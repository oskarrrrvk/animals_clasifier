\documentclass{article}

\usepackage[a4paper,total={18cm,25cm}]{geometry}
\usepackage[spanish]{babel}

\usepackage{graphicx}
\usepackage{makeidx}

\graphicspath{{components/}}

\title{Animals clasifier}
\author{Alejandro García Villalba, Sofía Díaz Fernandez, Óscar Herrero Gordaliza}

\begin{document}
    \maketitle
    \begin{figure}
        \centering
        \includegraphics[width=8cm]{politecnica_logo.png}
        \includegraphics[width=8cm]{etsisi_logo.png}       
    \end{figure}

    \newpage

    \tableofcontents
        \section{Contexto}
        \section{Arquitectura}
        \section{resultados}

    \newpage

    \title{Contexto}\\

    Animals clasifier es una IA creada para distinguir distintas especies de animales, entre ellos están: Perros,caballos, elenfantes, mariposas,
    gallinas, gatos,vacas, ovejas, arañas y ardillas.\\
    \begin{figure}
        \centering
        \includegraphics{dataset_images.png}
    \end{figure}
    Para entrenar esta neurona hemos escogido un dataset perteneciente a un concurso de kaggel llamado "mg-animal-prediction-24-25", creado este dataset
    como reto educativo para la asignatura de métodos generativos, como introducción al entrenamiento de redes de neuronas densas y convolucionales.
    Cabe añadir, con motivo de aprendizaje, este documento se redacta en latex como primer contacto para proyectos posteriores, por su versatilidad
    a la hora de diseñar la estructura de la documentación de manera mas flexible. 

\end{document}